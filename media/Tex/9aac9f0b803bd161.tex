\documentclass[preview]{standalone}

\usepackage[english,russian]{babel}
\usepackage[utf8]{inputenc}
\usepackage[T2A]{fontenc}
\usepackage{amsmath}
\usepackage{amssymb}
\usepackage{dsfont}
\usepackage{setspace}
\usepackage{tipa}
\usepackage{relsize}
\usepackage{textcomp}
\usepackage{mathrsfs}
\usepackage{calligra}
\usepackage{wasysym}
\usepackage{ragged2e}
\usepackage{physics}
\usepackage{xcolor}
\usepackage{microtype}
\DisableLigatures{encoding = *, family = * }
%\usepackage[UTF8]{ctex}
\linespread{1}

\begin{document}

\centering В выпуклом четырехугольнике $ABCD$ заключены две окружности одинакового радиуса $r$, 
                          касающиеся друг друга внешним образом. Центр первой окружности находится на отрезке 
                          соединяющем вершину $A$ с серединой $F$ стороны $CD$, а центр второй окружности находится 
                          на отрезке, соединящем вершину $C$ с серединой $E$ стороны $AB$. Первая окружность касается 
                          сторон $AB$, $AD$, $CD$; торая окружность касается сторон $AB$, $DC$, $CD$. Найти сторону $AC$.

\end{document}
